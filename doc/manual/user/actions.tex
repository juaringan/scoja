% A full description of each pure action, with its parameters.

\section{Pure Actions}

%----------------------------------------------------------------------
\subsection{Setting variables}
Event variables are defined and changed with {\tt set} action.
This action has two arguments;
the first one is the variable name;
the second one is a literal string or an Scoja expression.
This action is executed when the event pass through the link to its
successors.
The assignment takes effect until other {\tt set} is executed with the
same variable name or when leaving a previous executed {\tt local}
action.

Sometimes we want to restrict an assignment effect to a region of
a logging net; {\tt local} action can be used for this purpose.
When an event reach a {\tt local} link, a new subenvironment is built.
All assignments executed at its successors are stored in this
subenvironment.
When all {\tt local} successors are executed, the subenvironment is
deleted; so, all the successors assignments are reverted and previous
values are recovered.

In the following example
\begin{verbatim}
source >> ( local >> ( lognet1
                     | set("x", "xValue1") >> lognet2
                     | lognet3
                     | local >> set("x", "xValue2") >> lognet4
                     | lognet5
                     )
          | lognet6
          )
\end{verbatim}
variable {\tt x} has value {\tt "xValue1"} while Scoja is processing at
{\tt lognet2}, {\tt lognet3} and {\tt lognet5} logging subnets,
it has value {\tt "xValue2"} while processing {\tt lognet4},
but it is undefined while at {\tt lognet1} and {\tt lognet6}.

%----------------------------------------------------------------------
\subsection{Choosing timestamp reference}
Section~\ref{target.common.template} explains that every event has two
timestamps: send (remote) timestamp and Scoja (local) timestamp.
One of these is marked as \emph{default} timestamp.
When a new event is built, Scoja timestamp is marked as \emph{default}
timestamp.
Action {\tt useSendTimestamp} can be used to change which timestamp is
the default timestamp.
This action has a boolean argument;
if {\tt yes} o {\tt true}, send timestamp is marked as default;
if {\tt no} o {\tt false}, Scoja timestamp is marked as default.
This action it is executed when the event pass through the link to its
successors; it takes effect until other {\tt useSendTimestamp} is
executed.

In the following example,
\begin{verbatim}
template = "${DATE} ${HOST} ${PROGRAM}: ${MESSAGE}"
source >> ( filter1 >> toFile(name="log1", template=template)
          | filter2 >> useSendTimestamp(yes) >> toFile(name="log2", template=t)
	  | filter3 >> toFile(name="log3", template=t)
	  | filter4 >> useSendTimestamp(no) >> toFile(name="log4", template=t)
	  )
\end{verbatim}
events written to {\tt log1} and {\tt log4} use Scoja timestamp,
and events written to {\tt log2} and {\tt log3} use send timestamp.

