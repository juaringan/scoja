% A full description of each function, with its parameters.

\section{Functions}


%======================================================================
\subsection{Avoiding DoS with {\tt confine} function}
It is usual to split logs according to a value extracted from the
event.
Some sources for these values are secure: timestamp or priority.
But when using a literal string from the tag or the message, it is
easy to do a DoS attack to Scoja if no countermeasure is taken.

One posibility is to use {\tt mapping}; but the set of values must be
fixed and known when defining the configuration file.
Otherwise, {\tt confine} function is the best choice.
{\tt confine} restrict the number of different values that can happen
in a period of time.
It needs a first anonymous compulsory argument with the expression to confine.
A list with the optional tagged arguments follows:
\begin{description}
  \item[{\tt history}]
    How long a legal value is remembered.
    Optional; default value is one day.
  \item[{\tt periods}]
    {\tt history} is splitted in {\tt periods} periods.
    Optional; default value is 24.
  \item[{\tt max}]
    Maximum number of different values.
    Optional; default value is 10.
  \item[{\tt default}]
    A string with the value to produce when the limit of different
    values is exceeded.
    Optional; default value is {\tt CONFINED}.
\end{description}
