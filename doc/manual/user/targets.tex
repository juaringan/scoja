% A full description of each target, with its parameters.

\section{Targets}

%======================================================================
\subsection{Common elements}\label{target.common}

%----------------------------------------------------------------------
\subsubsection{Controling flushing}\label{target.common.flushing}
There are three flushing methods:
\begin{description}
  \item[{\tt BUFFER}]
    Data is send to the output (file, console, network) when internal
    buffers are filled.
  \item[{\tt FLUSH}]
    Data is send to the OS after each event.
  \item[{\tt SYNC}]
    Does {\tt FLUSH} and sync request to the OS.
\end{description}
These methods can be used directly as a legal value for {\tt flush}
attribute.

To {\tt FLUSH} or {\tt SYNC} only after $n$ events are send to the
target, it is necessary to use a full {\tt flushing} element.
This elements has the following attributes:
\begin{description}
  \item[{\tt method}]
    The flushing method, that can be {\tt BUFFER}, {\tt FLUSH}, or
    {\tt SYNC}.
    Optional; default value is {\tt FLUSH}.
  \item[{\tt after}]
    Defines after how many events a flushing occurs.
    Should be an interger greater than or equal to 1.
    Optional; default value is 1.    
  \item[{\tt buffer}]
    Size of internal buffer in bytes.
    If it is less than or equal to 0, no buffer is build.
    Optional; default value is 0.
\end{description}

Some examples.
To flush data to the OS after 10 events:
\begin{verbatim}
  flushing(method = FLUSH, after = 10)
\end{verbatim}
Because {\tt FLUSH} is the default method, previous flushing element
can be abbreviated to
\begin{verbatim}
  flushing(after = 10)
\end{verbatim}
To fully synchronize a file after 10 events
\begin{verbatim}
  flushing(method = SYNC, after = 10)
\end{verbatim}
Flushing elements are regular Python values;
it can be stored at a variable
or given directly as a value to {\tt flush} attributes:
\begin{verbatim}
  f1 = flushing(after = 10)
  target1 = toFile("/tmp/scoja/target1", flush = f1)
  target2 = toFile("/tmp/scoja/target2",
                   flush = flushing(method = SYNC, after = 10))
\end{verbatim}

%----------------------------------------------------------------------
\subsubsection{Controling file building}\label{target.common.building}
There are many details to configure when creating a new file.
Element {\tt building} serves this purpose.
It has the following attributes:
\begin{description}
  \item[{\tt fileOwner}]
    Specifies the file owner.
    If {\tt None}, process effective user is the file owner.
    Optional; default is {\tt None}.
  \item[{\tt fileGroup}]
    Specifies the file group.
    If {\tt None}, process effective user is the file owner.
    Optional; default is {\tt None}.
  \item[{\tt filePerm}]
    Specifies file permissions.
    If {\tt None}, file permissions is the effective user default.
    Optional; default is {\tt None}.
    
  \item[{\tt dirOwner}]
    Specifies owner for directories that has to be created
    because they don't exists when opening the file.
    If {\tt None}, process effective user is the directories owner.
    Optional; default is {\tt None}.
  \item[{\tt dirOwner}]
    Specifies group for directories that has to be created
    because they don't exists when opening the file.
    If {\tt None}, process effective group is the directories group.
    Optional; default is {\tt None}.
  \item[{\tt dirPerm}]
    Specifies permissions for directories that has to be created
    because they don't exists when opening the file.
    If {\tt None}, directories permissions are the effective user default.
    Optional; default is {\tt None}.
  
  \item[{\tt mkdirs}]
    Specifies whether directories should be created when opening a
    file whose path contains directories that doesn't exists.
    Optional; default is {\tt no}.
\end{description}

%----------------------------------------------------------------------
\subsubsection{Templates}\label{target.common.template}
Templates are the way to build strings out of event data.
They are mainly used to name files and to specify the format to store
events.

Templates are just strings; they have no special syntax.
But when a template is expected, \verb|$| character has a special
meaning.
Characters after \verb|$| are considered to be a variable name;
the template value will be computed substituting these variables for
their values.
There are predefined variable names that will be replaced by certain
event parts;
value of non-predefined variables will be looked for in the event
context.
Finally, to get a \verb|$| write \verb|$$|.

Before being used, templates are \emph{preprocessed}.
This preprocessing can be forced with the function {\tt template}.
Its result can be used wherever a template is required.

Usually variables are formed by alphanumeric characters;
the first non-alphanumeric character ends the variable.
For example, the following template
\begin{verbatim}
  "/tmp/scoja/$YEAR/$MONTH/$DAY/access.log"
\end{verbatim}
contains three (predefined) variables named 
  \verb|YEAR|, \verb|MONTH| and \verb|DAY|.
If we want to use non-alphanumeric characters at a variable name,
or if an alphanumeric characters follows after the variable,
the variable name should be surrounded with braces.
For instance
\begin{verbatim}
  "<${PRIORITY#}>${@DATE} ${HOST} ${PROGRAM}: ${MESSAGE}"
\end{verbatim}
Of course, braces can be used when not strictly necessary;
the following template is equivalent to the first one:
\begin{verbatim}
  "/tmp/scoja/${YEAR}/${MONTH}/${DAY}/access.log"
\end{verbatim}
We recommend to always use braces.

These are the predefined variables related to Scoja server
identification.
\begin{description}
  \item[{\tt MYIP}]
    This expands to the scoja server main IP.
  \item[{\tt MYHOST}]
    This expands to the scoja server name.
  \item[{\tt MYCHOST}]
    This expands to the scoja server full canonical name.
  \item[{\tt ME}]
    Expands to the value of the property {\tt org.scoja.server.name},
    if defined.
    Otherwise is equivalent to {\tt MYHOST}.
\end{description}

These are predefined variables not related to dates:
\begin{description}
  \item[{\tt FACILITY}]
    The name of the message facility.
  \item[{\tt LEVEL}]
    The name of the message level.
  \item[{\tt PRIORITY}]
    The name of the message priority.
    It is equivalent to \verb|${FACILITY}.${LEVEL}|.
  \item[{\tt FACILITY\#}]
    The message facility as a decimal number.
  \item[{\tt LEVEL\#}]
    The message level as a decimal number.
  \item[{\tt PRIORITY\#}]
    The message priority as a decimal number.
  
  \item[{\tt PROGRAM}]
    The name of the program that send the message.
  \item[{\tt MESSAGE}, {\tt MSG}]
    The message contents.
  \item[{\tt DATA}]
    The data part: the program, a colon and the message.
    It is roughly equivalent to \verb|${PROGRAM}: ${MESSAGE}|,
    but its exact value is extracted verbatim from the event.
    
  \item[{\tt IP}]
    The ip of the last host that send the message.
  \item[{\tt HOST}]
    The name of the first host that send the message.
    This data comes from the message but it can be absent;
    in this case, this variable behaves as \verb|${RHOST}|.
  \item[{\tt RHOST}]
    The host name of the last host that send the message.
    That is, the name computed with a reverse resolution of the IP.
  \item[{\tt CRHOST}]
    The preferred name for the last host that send the message.
    That is, the canonical name computed with a reverse resolution of the IP.
  \item[{\tt CHOST}]
    The preferred name for the first host that send the message.
    This is computed from the host in the message, or from the IP if
    the host is absent.
\end{description}

Two timestamps are attached to each message event:
one at the source host, when it is generated,
and another at Scoja, when it is received.
Both timestamps are useful but have some problems.
Source timestamp can be absent and can be facked.
Scoja timestamp isn't the real event date but cannot be facked and it
is the only way to build date-sorted files on the fly.
So, Scoja let us to choose which timestamp to use.
Every predefined variable related to time has three versions:
\verb|@X|, \verb|X@| and \verb|X|.
If there is no source timestamp, Scoja timestamp must be used and they
are equivalent.
The \verb|X@| version always use Scoja timestamp.
The \verb|@X| version tries to use source timestamp first.
The \verb|X| version uses the default timestamp;
when an event is build, default timestamp is Scoja timestamp, but this
default can be changed with action {\tt useSendTimestamp}.

Now we will list the default timestamp predefined variables.
Prepend (append) a \verb|@| to get the corresponding
source (Scoja) timestamp variable.
\begin{description}
  \item[{\tt DATE}, {\tt TIMESTAMP}]
    Standard syslog date format.
    It is not a good format because it has neither year nor
    milliseconds.
    Example: \verb*|Aug  2 08:40:49|.
  \item[{\tt SHORTDATE}, {\tt SHORTTIMESTAMP}]
    A full timestamp, with year but without milliseconds or GMT.
    Example: \verb*|2003-08-02 08:40:49|.
  \item[{\tt LONGDATE}, {\tt LONGTIMESTAMP}]
    A full timestamp, with year an with 3 digit millisecond, but
    without GMT.
    Example: \verb*|2003-08-02 08:40:49.123|.
  \item[{\tt GMTDATE}, {\tt GMTTIMESTAMP}]
    Like {\tt MILLISDATE} but with GMT. 
    Example: \verb*|2003-08-02 08:40:49.123 GMT+02:00|.   
    
  \item[{\tt YEAR}]
    All the digits of the year
  \item[{\tt \_YEAR}]
    The last two digits of the year
    
  \item[{\tt MONTH}]
    Two digits month number (01, \ldots, 12).
  \item[{\tt \_MONTH}]
    The month number (1, \ldots, 12).
  \item[{\tt MONTHNAME}]
    English full month name, variable length (January, \dots,
    December).
  \item[{\tt \_MONTHNAME}]
    English abbreviated month name (Jan, \dots, Dec).
    
  \item[{\tt DAY}]
    Two digits month day number (00, \ldots, 31).
  \item[{\tt \_DAY}]
    Month day number (0, \ldots, 31).
  
  \item[{\tt WEEKDAY}]
    English full weekday name, variable length (Sunday, \dots,
    Saturday).
  \item[{\tt \_WEEKDAY}]
    English abbreviated weekday name (Sun, \dots, Sat).
    
  \item[{\tt HOUR}]
    Two digits hour (00, \dots, 23).
  \item[{\tt \_HOUR}]
    An hour (0, \dots, 23).
    
  \item[{\tt MINUTE}, {\tt MIN}]
    Two digits minute (00, \dots, 59).
  \item[{\tt \_MINUTE}, {\tt \_MIN}]
    A minute (0, \dots, 59).
    
  \item[{\tt SECOND}, {\tt SEC}]
    Two digits second (00, \dots, 59).
  \item[{\tt \_SECOND}, {\tt \_SEC}]
    A second (0, \dots, 59).
  
  \item[{\tt MILLISECOND}, {\tt MILLIS}]
    Three digits millisecond (000, \dots, 999).
  \item[{\tt \_MILLISECOND}, {\tt \_MILLIS}]
    A millisecond (0, \dots, 999).
  
  \item[{\tt EPOCH}]
    Time as UTC seconds from the epoch.
  \item[{\tt MILLIEPOCH}]
    Time as UTC milliseconds from the epoch.
  
  \item[{\tt TZ}]
    Expands to the standard abbreviated time zone display name.
    It should not be used, because there are ambiguities.
  \item[{\tt TZOFFSET}]
    RFC 822 time zone.
    Format is
      $$\mbox{{\it Sign} {\it Hours2} {\it Minutes}}$$
    where 
      {\it Sign} is either \verb|+| or \verb|-|,
      {\it Hour2} is a 2 digits number between 0 and 23,
      {\it Minute} is a 2 digits number between 0 and 59.
  \item[{\tt GTZOFFSET}]
    General time zone.
    Format is 
      $$\mbox{{\tt GMT} {\it Sign} {\it Hours} {\tt:} {\it Minutes}}$$
    where
      {\it Sign} is either \verb|+| or \verb|-|,
      {\it Hour} is a 1 or 2 digits number between 0 and 23,
      {\it Minute} is a 2 digits number between 0 and 59.
\end{description}


%======================================================================
\subsection{Targets {\tt stdout} and {\tt stderr}}
Target {\tt stdout} (resp. {\tt stderr}) print log entries at the standard
output (resp. standard error).
Both have the same parameters:
%
\begin{description}
  \item[{\tt template}]
    Specifies the format for the event.
    It is a template (see~\ref{target.common.template}).
    
  \item[{\tt flush}]
    To describe the flushing details (see~\ref{target.common.flushing}).
\end{description}
%


%======================================================================
\subsection{Target {\tt toFile}}
This target write log entries in a file.
Templates are used to especify both filename and log entries format.
%
\begin{description}
  \item[{\tt name}] (Compulsory)
    File where events reaching this link are stored.
    It is a template (see~\ref{target.common.template}).
    
  \item[{\tt template}]
    Specifies the format for the event.
    It is a template (see~\ref{target.common.template}).
    
  \item[{\tt build}]
    Describes file building details (see~\ref{target.common.building})
    such as user, group and permissions.
    
  \item[{\tt flush}]
    Describes flushing details (see~\ref{target.common.flushing}).
\end{description}
%
